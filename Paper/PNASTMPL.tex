%%%%%%%%%%%%%%%%%%%%%%%%%%%%%%%%%%%%%%%%%
% Proceedings of the National Academy of Sciences (PNAS)
% LaTeX Template
% Version 1.0 (19/5/13)
%
% This template has been downloaded from:
% http://www.LaTeXTemplates.com
%
% Original author:
% The PNAStwo class was created and is owned by PNAS:
% http://www.pnas.org/site/authors/LaTex.xhtml
% This template has been modified from the blank PNAS template to include
% examples of how to insert content and drastically change commenting. The
% structural integrity is maintained as in the original blank template.
%
% Original header:
%% PNAStmpl.tex
%% Template file to use for PNAS articles prepared in LaTeX
%% Version: Apr 14, 2008
%
%%%%%%%%%%%%%%%%%%%%%%%%%%%%%%%%%%%%%%%%%

%----------------------------------------------------------------------------------------
%	PACKAGES AND OTHER DOCUMENT CONFIGURATIONS
%----------------------------------------------------------------------------------------

%------------------------------------------------
% BASIC CLASS FILE
%------------------------------------------------

%% PNAStwo for two column articles is called by default.
%% Uncomment PNASone for single column articles. One column class
%% and style files are available upon request from pnas@nas.edu.

%\documentclass{pnasone}
\documentclass{pnastwo}

%------------------------------------------------
% POSITION OF TEXT
%------------------------------------------------

%% Changing position of text on physical page:
%% Since not all printers position
%% the printed page in the same place on the physical page,
%% you can change the position yourself here, if you need to:

% \advance\voffset -.5in % Minus dimension will raise the printed page on the 
                         %  physical page; positive dimension will lower it.

%% You may set the dimension to the size that you need.

%------------------------------------------------
% GRAPHICS STYLE FILE
%------------------------------------------------

%% Requires graphics style file (graphicx.sty), used for inserting
%% .eps/image files into LaTeX articles.
%% Note that inclusion of .eps files is for your reference only;
%% when submitting to PNAS please submit figures separately.

%% Type into the square brackets the name of the driver program 
%% that you are using. If you don't know, try dvips, which is the
%% most common PC driver, or textures for the Mac. These are the options:

% [dvips], [xdvi], [dvipdf], [dvipdfm], [dvipdfmx], [pdftex], [dvipsone],
% [dviwindo], [emtex], [dviwin], [pctexps], [pctexwin], [pctexhp], [pctex32],
% [truetex], [tcidvi], [vtex], [oztex], [textures], [xetex]

\usepackage{graphicx}

%------------------------------------------------
% OPTIONAL POSTSCRIPT FONT FILES
%------------------------------------------------

%% PostScript font files: You may need to edit the PNASoneF.sty
%% or PNAStwoF.sty file to make the font names match those on your system. 
%% Alternatively, you can leave the font style file commands commented out
%% and typeset your article using the default Computer Modern 
%% fonts (recommended). If accepted, your article will be typeset
%% at PNAS using PostScript fonts.

% Choose PNASoneF for one column; PNAStwoF for two column:
%\usepackage{PNASoneF}
%\usepackage{PNAStwoF}

%------------------------------------------------
% ADDITIONAL OPTIONAL STYLE FILES
%------------------------------------------------

%% The AMS math files are commonly used to gain access to useful features
%% like extended math fonts and math commands.

\usepackage{amssymb,amsfonts,amsmath}

%------------------------------------------------
% OPTIONAL MACRO FILES
%------------------------------------------------

%% Insert self-defined macros here.
%% \newcommand definitions are recommended; \def definitions are supported

%\newcommand{\mfrac}[2]{\frac{\displaystyle #1}{\displaystyle #2}}
%\def\s{\sigma}

%------------------------------------------------
% DO NOT EDIT THIS SECTION
%------------------------------------------------

%% For PNAS Only:
\contributor{Submitted to Proceedings of the National Academy of Sciences of the United States of America}
\url{www.pnas.org/cgi/doi/10.1073/pnas.0709640104}
\copyrightyear{2008}
\issuedate{Issue Date}
\volume{Volume}
\issuenumber{Issue Number}

%----------------------------------------------------------------------------------------

\begin{document}

%----------------------------------------------------------------------------------------
%	TITLE AND AUTHORS
%----------------------------------------------------------------------------------------

\title{Multifactor, multiple people. Authentication approach for unlocking encrypted fies.} % For titles, only capitalize the first letter

%------------------------------------------------

\author{Yuri Gorokhov\affil{1}{University of California, San Diego},
Lars Noergaard Nielsen\affil{1}{University of California, San Diego}}

\contributor{Submitted as part of graduate security course CSE227.}

%----------------------------------------------------------------------------------------

\maketitle % The \maketitle command is necessary to build the title page

\begin{article}

%----------------------------------------------------------------------------------------
%	ABSTRACT, KEYWORDS AND ABBREVIATIONS
%----------------------------------------------------------------------------------------

\begin{abstract}
Proposal of a system to authenticate access to encrypted files using both Multifactor and multiple people, across different locations. Making sure that files are only accessible with the consent of all involved participants.
\end{abstract}

%------------------------------------------------

\keywords{Multifactor | Encryption} % When adding keywords, separate each term with a straight line: |

%------------------------------------------------

%----------------------------------------------------------------------------------------
%	PUBLICATION CONTENT
%----------------------------------------------------------------------------------------

%% The first letter of the article should be drop cap: \dropcap{} e.g.,
%\dropcap{I}n this article we study the evolution of ''almost-sharp'' fronts

\section{Introduction}

\dropcap{C}ontrolling who has access to files is often a requirement in industry and various other contexts. Systems for dealing with information that only is accessible with multiple people's consent is therefore interesting to investigate. Software for file access control purposes include Dell Identity Manager\cite{CLAcha1}, User Lock Access Manager \cite{UserLock} and native OS support such as an Access Control List. These systems is not addressing security as such, as not providing encryption capabilities. Common for these solutions is that file access is administered centrally by a administrator. We propose an approach were users actively set file permissions by agreeing to encrypt files by their common consent, only allowing access to these files when all parties have responded to the access request. The latter step is additionally secured by MultiFactor Authentication.

\section{Yubikey}
TODO

\section{Previous work}
We based our model on a proposed hard drive encryption mechanism published on the Yubikey website\cite{YubikeyEncryption}. In their proposed configuration the Yubikey is programmed with a secret key after which it is able to perform HMAC-SHA1 encryption. The device is said to be operating in Challenge-Response mode since you can send it a challenge and it will respond with the HMAC-SHA1 encryption of the challenge with the secret key. This is depicted in figure 1.


\begin{figure}
\centerline{\includegraphics[width=\linewidth]{ChallengeResponseFigure.pdf}}
\caption{Challenge-Response mode of Yubikey.}\label{systemfigure}
\end{figure}

Figure 2 shows how this mode is used for encrypting a local hard drive. The hard drive is encrypted with a Drive Encryption Key (DEK) which is stored in a table along with the secret. Both are encrypted using AES. The key used for the encryption is shown in Figure 2. Once the user enters his password a challenge is generated. The challenge consists of the password itself and a sequence number (Seq) that is also stored in the table. This challenge is sent to the Yubikey, whose response allows us to decrypt the DEK and the secret. The hard drive can now be decrypted. After decryption the DEK is re-encrypted with a new sequence number and the secret.

\begin{figure}
\centerline{\includegraphics[width=\linewidth]{HardDriveEncryptionFigure.pdf}}
\caption{Local hard drive encryption configuration.}\label{systemfigure}
\end{figure}

%------------------------------------------------

\section{System Description}

The proposed system is composed of a user client, harddrive client and a verification server. 

A benefit from this central point of contol is that the server can deny access to a file, even though participants grant access, which might be useful in some access schemes.

\begin{figure}
\centerline{\includegraphics[width=\linewidth]{SystemFigure.pdf}}
\caption{System components and interaction. 4 major steps to grant access.}\label{systemfigure}
\end{figure}

\section{Multifactor authorization: Yubikey}

In this work we choose Yubikey for Multifactor authentication for several reasons: It provides a simple procedure for the user - only a physical touch on the device is neccesary to allow the device to respond to the presented challenge.


\section{Encryption scheme}

This sections presents the encryption scheme used, to ensure that decryption of the file is only possible when responses from all participants and their Yubikey challenges are retrieved.

%------------------------------------------------

\section{Discussion}

Discussion on strenghts and weaknesses of the solution

%----------------------------------------------------------------------------------------
%	ACKNOWLEDGEMENTS
%----------------------------------------------------------------------------------------

\pagebreak

\begin{acknowledgments}
This work was supported by..
\end{acknowledgments}

%----------------------------------------------------------------------------------------
%	BIBLIOGRAPHY
%----------------------------------------------------------------------------------------

\begin{thebibliography}{10}


\bibitem{UserLock}
http://www.isdecisions.com/lp/userlock/userlock-windows-network-security.htm?gclid=CMfPl-rqisQCFciBfgodhxwAmQ

\bibitem{CLAcha1}
http://software.dell.com/products/identity-manager-data-governance/

\bibitem{YubikeyEncryption}
https://www.yubico.com/applications/disk-encryption/full-disk-encryption/

\end{thebibliography}

%----------------------------------------------------------------------------------------

\end{article}

\end{document}